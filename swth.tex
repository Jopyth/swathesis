\documentclass{scrartcl}
\usepackage{swa-platform}
\usepackage{swa-language,swa-document-koma}
\usepackage{textcomp}
\usepackage{textcase}
\usepackage{relsize}
\usepackage[final,babel=true]{microtype}
\ifboolexpr{bool{tutex}}%
  % yes otf
  {\usepackage{fontspec}
    \microtypesetup{stretch=9,shrink=15,step=3}
    \setmainfont[ExternalLocation,%
      Numbers={Proportional,OldStyle},%
      UprightFont= *-regular,%
      BoldFont=*-bold,%
      ItalicFont=*-italic,%
      BoldItalicFont=*-bolditalic,%
      Ligatures={Common,TeX},%
      ]{texgyrepagella}
    \setmonofont[ExternalLocation,
      Scale=MatchLowercase,%
      \ifboolexpe{bool{luatex}}%
        {RawFeature={-tlig;-trep}}%
        {Mapping=},% empty, no ligs
      BoldFont={*-Bold},
      ItalicFont={*-Oblique},
      BoldItalicFont={*-BoldOblique}
    ] {DejaVuSansMono}}%
  % no otf
  {\usepackage{fontaxes}
    \microtypesetup{stretch=9,shrink=15,step=3,tracking=smallcaps,letterspace=75}
    \usepackage[largesc,p,osf]{newpxtext}
    \usepackage[scaled=.84]{DejaVuSansMono}
    \usepackage{newpxmath}}
\linespread{1.05} % a bit more for Palatino
\usepackage{verbatim}

\usepackage[nofancy]{latex2man}
\renewcommand{\Prog}[1]{\texttt{#1}}                      % Program name
\renewcommand{\Cmd}[2]{\texttt{#1}(#2)}                   % Command with number
\makeatletter
\renewcommand{\@LM@Arg}[1]{\texttt{\textit{#1}}}
\renewcommand{\@LM@Opt}[1]{\texttt{\textbf{#1}}}
\makeatother
\usepackage{scrlayer-scrpage}
\pagestyle{scrheadings}
\begin{document}
\setdescription{noitemsep,font=\normalfont,leftmargin=\parindent,topsep=.5\baselineskip,partopsep=.5\baselineskip}
\setDate{2021-10-01}
\setVersion{1.0}

\begin{Name}{1}{swth}{Tobias Pape}{swth \Version}{swth}
  \Prog{swth} -- create or manage \File{swathesis} documents.
\end{Name}

\section{Synopsis}

\Prog{swth}
  \oOpt{options}
  \Arg{command}

\section{Description}

\Prog{swth} is a tool to manage documents using the swathesis bundle

\section{Options}

\begin{Description}[\Opt{--bachelor}]\setlength{\itemsep}{0pt}
\item[\Opt{--bachelor}] operate in BSc mode
\item[\Opt{--master}]   operate in MSc mode
\item[\Opt{--phd}]      operate in PhD mode
\end{Description}

\begin{Description}[\Opt{--bachelor}]\setlength{\itemsep}{0pt}
\item[\Opt{--xe}]       use Xe\LaTeX
\item[\Opt{--pdf}]      use pdf\LaTeX\ (default)
\item[\Opt{--lua}]      use Lua\LaTeX
\item[\Opt{--biber}]    use Biber instead of BibTeX
\end{Description}

\begin{Description}[\Opt{--bachelor}]\setlength{\itemsep}{0pt}
\item[\Opt{--help}]     output this help and exit
\item[\Opt{--dry-run}]  don't execute, just say what would be
\item[\Opt{--version}]  output version information and exit
\end{Description}

\section{Commands}

\begin{Description}[\Arg{author}]\setlength{\itemsep}{0pt}
\item[\Arg{create}]     create swathesis document
\item[\Arg{init}]     put directory under \Prog{swth} management
\end{Description}

\begin{Description}[\Arg{author}]\setlength{\itemsep}{0pt}
\item[\Arg{bibtex}]     run bibtex [especially for BSc mode]
\item[\Arg{latex}]     run latex (honors \Opt{options} and management file)
\item[\Arg{gloss}]     run makeglossaries [especially for PhD mode]
\item[\Arg{show}]     open result document
\item[\Arg{clean}]     clean auxiliary files
\item[\Arg{go}]     short for latex; bibtex; latex; latex; latex; show
\end{Description}

\begin{Description}[\Arg{author}]\setlength{\itemsep}{0pt}
\item[\Arg{author}]     add another author in BSc mode
\end{Description}

\section{Files}

\begin{Description}\setlength{\itemsep}{0pt}
\item[\File{.swth}] Controls the master file for \LaTeX operations and certain
  variables. \Prog{swth} will search up to three levels of parent directories
  for such a file.
\end{Description}

\section{Bugs}

Plenty. Beware of Dragons.

\section{Examples}

\subsection{Create a new document}

\verb+swth --lua --phd --biber create+\\
asks for directory/name/title and create template files, as well as
\File{.swth} file

\subsection{Run \LaTeX}

\verb+swth latex+\\
run current latex command on current main file, as defined in \File{.swth} file

\subsection{Everything}

\verb+swth go+\\
run everything on current main file, as defined in \File{.swth} file, showing
pdf at the end

\section{See Also}

\Cmd{latexmk}{1}, \Prog{arara}

\end{document}
%%% Local Variables:
%%% mode: latex
%%% TeX-master: t
%%% coding: utf-8
%%% End:
